\documentclass{article}
\usepackage{amsmath}
\author{Zixuan Guo}
\begin{document}

\section{Problem 1}

\paragraph{}
In this problem, we choose the comparsion as the relevant operation to count. Under the worst-case scenario, we needs to traverse
the entire array to conclude if the double array contains the number we are looking for. 
Therefore, for an array with n elements, we have
\[T(n) = n\]
as number of operations for double array with given length.

\paragraph{}
We conclude $ T(n) \in O(n) $ because the inequality $ \mid f(n) \leq c\mid g(n) \mid\mid $ holds when $ c = 1 $ and $ n_0 = 0 $,
since $ f(n) \leq O(n) $ for $ n \geq 0 $.


\section{Problem 2}

\paragraph{}
We find the pattern of the fastModExp method as:
\[ x^y = \begin{cases} 
    1 \text{ mod } m &x = 0 \\
    (x^2 \text{ mod } m)^\frac{y}{2} \text{ mod } m &\text{when y is even} \\
    (x * (y^{(y-1)} \text{ mod } m)) \text{ mod } m &\text{when y is odd}
\end{cases} \]
Therefore, we define all arithmetic operation as relevant operation and assume all of them is $ O(1) $ operation.
Then, we obtain the following recurrence for fastModExpt:
\[ T(y) = T([y/2]) + 2\]
where we assume the input y is a power of two. 
And this recurrence implies the closed-form solution $ T(y)=2*O(\log y) + 3$ if we draw the recursion tree.

\paragraph{}
We conclude $ T(y) \in O(\log n) $. To see this, we node that, for $ n \geq 2 $, we have $ T(n) \leq 3*\log_2 n $.
Therefore we have $ c = 3 $ and $ n_0 = 2 $.


\section{Problem 3}

\paragraph{}
In this problem, we consider array access as the relevant operation. Then we can obtain the model 
\[ T(n) = 3n^2 \]
where n is the number of element of the input array.

\paragraph{}
Then we can conclude that $ T(n) \in O(n^2) $. To see this, note that for $ n \geq 0 $, $ T(n) \leq 3 * n^2 $. 
Therefore we have $ c = 3 $ and $ n_0 = 0 $.


\section{Problem 4}

\subsection{Constant-time String Concatenation}
\paragraph{}
In this case, we define string concatenation as the relavent operation. We can write a model T of the time coplexity as 
\[ T(n, m) =  nm \]
where n is the n is the numbers of repetition and m is the number of strings in the input array.
We conclude $ T(n, m) \in O(n^2) $ because the program contains two loops and number of loops decides by both n and m.

\subsection{Linear-time String Concatenation}
\paragraph{}
For the sake of the simplity, we assume all the string have the same length $ l $ and we believe such simplication will not influence
the result of analysis. Then, we can translate the loop bounds to summations bounds: 
\[ \sum_{i=1}^{mn} il \]
and we can transform above form into
\[ T(m, n) = \frac{mn(mnl + l)}{2}\]

\paragraph{}
Therefore, we conclude that $ T(m,n) \in O(n^4) $ if we expand the formula above. The run-time of the program change from
$ O(n^2) $ to $ O(n^4) $ because the cost of concatenation changes from $ O(1) $ to $ O(n) $.


\section{Problem 5}

\subsection{Time Complexity}
\paragraph{}
For the analysis of time complexity, we decide to choose array access as relevant operation and then we have: 
\[ T(n,m)= m+n\]
where $ n $ and $ m $ is the number of elements in each of the input arrays.
Then we can conclude $ T(n,m) \in O(n) $ because the number of array access increase linearly as we increase the
length (number of elements) of input array.

\subsection{Space Complexity}
\paragraph{}
The space complexity of the function can be described by:
\[ T(m,n) = m + n \]
where $ n $ and $ m $ is the number of elements in each of the input arrays.
And we conclude that $ T(n,m) \in O(n) $ for space complexity the length of the new array we produce is the sum of two input array.

\subsection{Review}
\paragraph{}
We found that there is not any connection between space complexity and time complexity in genreal.
However, under certain scenario, we can establish a connection between them. For example, when the critical operation cost both extra time and space to perform.
\end{document}


